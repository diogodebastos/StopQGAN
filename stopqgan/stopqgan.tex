\documentclass[12pt, letterpaper]{article}

\usepackage{graphicx}
\usepackage{pythonhighlight}
\usepackage[useregional]{datetime2}
\usepackage{glossaries}
\usepackage[english]{babel}
\usepackage{ragged2e}
\usepackage{blindtext}
\usepackage[italic,italicGreek]{heppennames2}
\usepackage{url}
\usepackage{biblatex}
\bibliography{stopqgan.bib}
%\usepackage[square,numbers,sort&compress]{natbib}
%\usepackage{biblatex}


\newcommand{\stp}{\ensuremath{\PSQt_{1}}\xspace}
\newcommand{\stpb}{\ensuremath{{\PASQt}}_{1}\xspace}

\makeglossaries
\loadglsentries{acronyms.tex}

\begin{document}
%\setlength{\hsize}{0.9\hsize}

%%% TITLE

\pagenumbering{gobble}

%\maketitle
\vspace{30mm}
\centering
\LARGE \textbf{A practical use case for Quantum Generative Adversarial Networks 
in High Energy Physics}\\
\vspace{5mm}
\large Generating top squark events decaying via the four-body mode in 
single-lepton final states in proton-proton collisions at $\sqrt{s}=13$ TeV\\
\vspace{15mm}
\Large \textbf{Diogo Carlos Chasqueira de Bastos} \\
\vspace{15mm}
\centering
\large \textbf{Advanced Topics in Particle and Astroparticle Physics II} \\
\date{\today}
\vspace{15mm}

\begin{abstract}
This is a simple paragraph at the beginning of the 
document. A brief introduction about the main subject.
\end{abstract}

%%% ROMAN

\clearpage
\setcounter{page}{1} \pagenumbering{roman}
\tableofcontents
\clearpage
\listoffigures
\clearpage
\listoftables
\clearpage
\printglossary[type=\acronymtype]
\clearpage

%%% ARABIC

\pagenumbering{arabic} \setcounter{page}{1}
\justifying

\section{Introduction}
\label{sec:intro}

One of the main objectives of \gls{qti} is to investigate if \gls{qc} can be 
used in the field of \gls{hep} in the \gls{nisq} era. Noisy means that, currently,
we have imperfect control over qubits. Intermediate refers to the number of qubits
our present quantum computers have. They range from 50 to a few hundred. 
In order to fully fulfill the promise of quantum computing, the challenges of
noise and scalability (millions of qubits) needs to be solved. Nonetheless, we
can already use the available quantum computers and algorithms to tackle present
challenges. 

\gls{qml} is a growing research area that explores the interplay of ideas from \gls{qc} 
and machine learning. This project explores the use of \glspl{qgan}~\cite{Zoufal_2019}, 
a \gls{qml} algorithm, to learn the kinematic distributions of stop four-body 
decays~\cite{ph-brief-stop} in \gls{cms} data. the Feynman diagram for such process 
is represented in Figure~\ref{fig:model}. In \gls{hep}, to simulate such 
distributions, \gls{mc} simulations are used in a very convoluted and computational
resources hungry process that can take up months. As we will see, the method 
presented in this project could speed up this process significantly when run on 
quantum computers. This method could also be used for data augmentation of the 
\gls{mc} generated samples which would be helpful in the training of the 
classification algorithms in many \gls{hep} searches improving the sensitivity of
such searches.

\clearpage

\begin{figure}[!htbp]
\centering
\includegraphics[width=0.70\textwidth]{figures/Figure_001.pdf}
\caption{Diagram of top squark pair production $\stp \stpb$ in $\Pp\Pp$ collisions, 
with a four-body decay of each top squark.}
\label{fig:model}
\end{figure}

This project will be run on a classical computer (my personal laptop) simulating
a quantum computer with 5 qubits using pennylane~\cite{pennylane} to train and 
optimize the quantum algorithm, and cirq~\cite{cirq} for writing, manipulating, 
and running the quantum simulator.

\clearpage
%\addcontentsline{toc}{part}{Bibliography}
%\bibliography{stopqgan}{}
%\bibliographystyle{plain}
%\bibliographystyle{IEEEtranN}
%\bibliographystyle{unsrt}
\printbibliography
\clearpage

\end{document}