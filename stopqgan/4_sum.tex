\section{Summary and Discussion}
\label{sec:sum}

Top squark events decaying via the four-body mode in single-lepton final states 
in proton-proton collisions at $\sqrt{s}=13$ TeV are generated using \glspl{qgan}
that run on a simulated quantum computer. 

This technique can be used to help solving the problem of limited statistics in 
\gls{mc} generated samples. It is important to note that this project serves as 
a proof of concept and that it is limited by being run on a simulated quantum 
computer. If run on a quantum computer, the next steps would include the increase
of qubits, to encode more kinematic variables, a speed up on the training time, 
the parallelization of the training for each event and finally, the collection 
of all the generated states in order to generate the full kinematic distributions. 
Given the novelty of this project and the possible use case, after the completion
of the latter steps, it would be interesting to publish the results. To do so, 
the main road blocker is the access to a quantum computer. This could be done
via LIP or LIP-CMS by requesting computing time to Xanadu, IBM, IonQ or Google AI
through \gls{qti} which has already established partnerships with the 
aforementioned companies.