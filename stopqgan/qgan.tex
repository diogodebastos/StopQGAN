QGANs are fundamentally composed of two Quantum Neural Networks (QNNs) - a generator and a discriminator one - that compete between each other. The generative network generates candidates while the discriminative network evaluates them. The contest operates in terms of data distributions. Typically, the generative network learns to map from a latent space to a data distribution of interest, while the discriminative network distinguishes candidates produced by the generator from the true data distribution. The generative network's training objective is to increase the error rate of the discriminative network (i.e., "fool" the discriminator network by producing novel candidates that the discriminator thinks are not synthesized (are part of the true data distribution)).

\begin{python}
def f(x):
    return x
\end{python}